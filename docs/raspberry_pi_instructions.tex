\documentclass{article}
\author{Oliver Kisielius}
\title{Problem Set SPECIFY HERE}
\usepackage{amsmath}
\usepackage{amssymb}
\usepackage{amsthm}
\usepackage{mathtools}
\usepackage{enumitem}

%% TODO: Compile this to a plain text file with catdvi:
%% http://catdvi.sourceforge.net/
%% Also, make sure to separate the parts of this document that do and don't
%% relate to this particular piece of software.

%% outline
%% 	Tell them what the goal is
%% 		setup raspberry pi to take measurements from load cell
%% 		take measurements
%% 		transfer data to USB drive
%% 	download Raspbian Jessie Lite from
%% 		https://www.raspberrypi.org/downloads/raspbian/
%% 	(In a few years, the name may be "Raspbian Stretch Lite" or similar.
%% 	Such is progress.)
%% 	Follow instructions here to install Raspbian Jessie Lite:
%% 		https://www.raspberrypi.org/learning/software-guide/quickstart/
%% 		(You may not need to install the Lite version, but it may work
%% 		better.)

\def\url{\texttt}
\newcommand{\ls}{\texttt{ls}}
\newcommand{\cd}{\texttt{cd}}
\begin{document}

This is my guide to setting up the Raspberry Pi and collecting data. The scope of
the guide is limited, but there are many free online resources that can cover the
gaps. Much of the Setup part of the guide simply lists those resources.

\subsection*{Preparing the microSD card}

You will need to format the microSD card with an image of the Raspbian operating
system. This should be available from the Raspberry Pi Foundation for the
forseeable future. At the time of writing, this page has it:\\
\url{https://www.raspberrypi.org/downloads/raspbian/}\\


There are two choices. The only difference is that the ``Lite'' version has fewer
features by default. While I used the Lite version when I ran the tests myself,
you will likely prefer to have the graphical interface that comes with the regular
version. All my instructions were written with Raspbian Lite in mind, which means
they will work just as well if you install the full version. When you follow my
instructions, don't use the \texttt{startx} command.

%% \begin{enumerate}
%% \item The ``Lite'' version is what I used to run the tests myself.
%%   By default, Raspbian Lite doesn't have any graphical programs installed,
%%   which may be irritating. Although I will provide some command line help,
%%   some tasks may still be easier if you have graphical tools installed.

%%   However, if you do choose Raspbian Lite, you can always change your mind and
%%   install the software that you desire. (This is straightforward but is still
%%   outside the scope of this guide.)
%% \item The normal version might be easier. Tasks such as moving data to a USB drive
%%   will be easier if you use the graphical mode. If you have a computer science
%%   major working for you, they should totally use the Lite version.
%% \end{enumerate}

Set up your 'Pi with a USB keyboard and an HDMI monitor, and turn it
on.  This is easier than it may sound. The Raspberry Pi Foundation has
many resources on their website to help you. For example, this one is
written for parents - but I love the diagram at the bottom:
https://www.raspberrypi.org/learning/parents-guide/setup/
(You won't need a USB mouse for Raspbian Lite. However, you will want
to connect to the internet.)

%%%%%%%%%%%%%%%%%%%%%%%%%%%%%%%%%%%%%%%%%%%%%%%%%%%%%%%%%%%%%%%%%%%%%%%%%%%%%%%%

\subsection*{Background}

As the Raspberry Pi guides will tell you, the 'Pi is just a computer.
If you've seen
\emph{Jurassic Park}, you may remember the scene where, in the midst of a crisis,
the young hacker girl approaches a computer and exclaims, "It's a UNIX system!
I \emph{know} this!" Well, the Raspberry Pi is a UNIX system.\footnote{Technically,
  it's a UNIX-\emph{like} system. But that doesn't really matter. Also, that
  movie scene is wildly inaccurate.}
It isn't very complicated. The files are organized the same way as on your
Windows computer: Everything is organized in folders (``directories''). Directories
can contain files, such as this PDF, and they can also contain other directories.
For example, your ``home directory'', which is named with your username,
probably contains subdirectories like \texttt{Desktop} and \texttt{My Documents}.
Linux also has a ``home directory'' for each user. Yours will be
\texttt{/home/pi}.

When you log in, you'll start out ``in'' your home directory.
The gobbledygook in front of your cursor doesn't mean anything - it's just
part of the command. It probably looks something like \texttt{pi@raspbian:~/ \$}.
If a problem arises and you search for an answer online, they may tell you to
enter commands like \texttt{~\$ command}. Only type in the part after the
\texttt{\$}.

The two important commands to know are \ls and \cd.
\ls lists the contents of the current directory. It might stand for ``List Stuff''
or ``Let's See what's going on, here''.\\
\cd means ``change directories''. This is just like double-clicking on a folder
in Windows Explorer.\\

\subsection*{A few irritating steps}

Once you turn on your Pi, there are just
three small steps to take before you start setting up the measurement software:

begin{enumerate}
\item Connect to the internet. The easiest way to do this is to connect the pi
  to an ethernet port. The pi should automatically connect to the internet
  if you do this.\\
  However, to make the connection actually work, you'll probably need to
  register the device with the university IT department.\footnote{This
    is not the poor Raspberry Pi's fault.}
  The IT people in the library will need to help you with this.\\
  They wanted me to register my Pi through the form on this webpage:\\
  \texttt{https://bradford.tntech.edu/registration/GameRegister.jsp}\\
  Make sure the URL starts with http\textbf{s} before you enter your password.
  For the ``device type'', select ``other.''\\
  They will want the MAC address of the Pi. To find this address, use the command
  \texttt{ip addr | grep -A2 'eth0'}. You'll see a subheading like \texttt{eth0},
  and under that subheading you will see a line beginning with \texttt{link/ether}.
  Directly to the right of \texttt{link/ether} will be the MAC address of the Pi.
  It will look something like \texttt{12:34:56:78:9a:bc}. Ignore the number that's
  all \texttt{ff:ff:$\dots$:ff}.\\
  If that sounds impossible, you can try typing in this enormous command instead:\\
  \texttt{ip addr | grep -A2 'eth0' | grep 'link/ether' | sed 's/.*ether
    \backslash([[:xdigit:]\backslash:]\{3,\}\backslash).*$/\backslash1/'}\\
  Of course, if neither of those things work, google it or ask somebody for help.

  %% The command I'm trying to portray here is this:
  %% ip addr | grep -A2 'eth0' | grep 'ether' | sed 's/.*ether \([[:xdigit:]\:]\{3,\}\).*$/\1/'
  %% TODO make sure LaTeX didn't interpret any of those characters specially.

\item Change the passphrase! By default, anybody who knows your passphrase can log
  into your 'Pi from anywhere in the world. And they certainly will. Remember that
  hackers automate everything they do. Without any human involvement, a program
  written by a hacker will detect your raspberry pi, attempt to login by trying all
  6 million passwords stolen from LinkedIn users last year, and do something
  scary like delete all your data. That's why the passphrase should be at least
  20 letters long and \emph{surprising}, like
  \texttt{kisielius masquerades as a competent cyber tyrannosaur} or
  \texttt{carillon phantom croons Summer Lovin' again}.\\
  To change your password on the 'Pi, enter the command \texttt{passwd}.
  You might want to write the password down somewhere in the lab.\footnote{By
    the way, make sure to backup the important data from
    your computer as often as possible. There is a new type of virus
    that steals all your files and tries to ransom them back to you,
    and it's becoming more and
    more common worldwide. More than half of the businesses in the UK
    were affected by it last year:
    \texttt{http://arstechnica.com/security/2016/08/more-than-half-of-uk-firms-have-been-hit-by-ransomware-report/}
  }

  For extra assurance that nobody will sign in with your password from afar,
  you can do this to edit the file \texttt{/etc/ssh/sshd_config}:\\
  Enter the command \texttt{sudo nano /etc/ssh/sshd_config}\\
  This opens the relevant file in a user-friendly text editor called
  \texttt{nano}.\\
  Use the arrow keys to move down to the line that says\\
  \texttt{PasswordAuthentication yes}\\
  and change the \texttt{yes} to \texttt{no}.\\
  Save your changes by pressing \texttt{CONTROL-o} and \texttt{ENTER}, and
  then exit with \texttt{CONTROL-x}.
  Now restart the 'Pi by entering \texttt{sudo reboot}.


\item Update the pi. This is easy. Just enter the commands:\\
  \texttt{sudo apt-get update}\\
  \texttt{sudo apt-get dist-upgrade}\\

  If this does not work, it is probably because the Pi is not connected to the
  internet properly.

\end{enumerate}

\section*{Set up the test.}

For this part, please follow the instructions in the other document.

\section*{Move data to a USB flash drive.}

Naturally, the first step is to plug the USB drive into one of the USB ports on
the 'Pi. Then you may need to ``mount'' the USB drive, which means ``tell your
computer to start talking to the USB drive, and make the USB drive act like a
normal folder on your computer so that you can move files to it.''

This article with the evocative name ``Raspberry Pi Mount a USB Drive Tutorial'',
is quite nice and even has a video. You probably won't need to read the whole
article.\\
\url{https://pimylifeup.com/raspberry-pi-mount-usb-drive/}\\

I recommend using the graphical mode to move data to the USB drive once the
drive is mounted.

\end{document}
